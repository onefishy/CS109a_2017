\documentclass[12pt,t]{beamer}
\usepackage{graphicx}
\usepackage[vlined]{algorithm2e}
\usepackage{times}
\usepackage{calc}
\usepackage{url}
\usepackage{soul}
\usepackage{graphicx}
\usepackage{multirow, hhline}
\usepackage{array, booktabs}
\usepackage{amsmath}
\usepackage{amssymb}
\usepackage{relsize}
\usepackage{multirow}
\usepackage{booktabs}
\usepackage{pagecolor}
\usepackage{lipsum}
\usepackage{capt-of}
\usepackage{booktabs}

\usepackage{graphicx}
\usepackage{multicol}
\usepackage[T1]{fontenc}
\usepackage{ae}
\graphicspath{{fig/}}
\setbeameroption{hide notes}
\setbeamertemplate{note page}[plain]

\usetheme{default}
\beamertemplatenavigationsymbolsempty
\hypersetup{pdfpagemode=UseNone}

\usefonttheme{professionalfonts}
\usefonttheme{serif}
\usepackage{fontspec}
\setmainfont{Karla}
\setbeamerfont{note page}{family*=pplx,size=\footnotesize} % Palatino for notes

\definecolor{foreground}{RGB}{70,70,70}
\definecolor{background}{RGB}{249, 249, 249} %24,24,24
%\definecolor{title}{RGB}{107,174,214} %107,174,214
\definecolor{title}{RGB}{70,70,70}
\definecolor{gray}{RGB}{0,0,0}
\definecolor{subtitle}{RGB}{70,70,70}
\definecolor{hilight}{RGB}{102,255,204}
\definecolor{vhilight}{RGB}{255,111,207}

\setbeamercolor{titlelike}{fg=title}
\setbeamercolor{subtitle}{fg=subtitle}
\setbeamercolor{institute}{fg=gray}
\setbeamercolor{normal text}{fg=foreground,bg=background}


\setbeamercolor{item}{fg=foreground} % color of bullets
\setbeamercolor{subitem}{fg=gray}
\setbeamercolor{itemize/enumerate subbody}{fg=gray}
\setbeamertemplate{itemize subitem}{{\textendash}}
\setbeamerfont{itemize/enumerate subbody}{size=\footnotesize}
\setbeamerfont{itemize/enumerate subitem}{size=\footnotesize}

\setbeamercolor{block title}{fg=white,bg=gray!70}
\setbeamercolor{block body}{fg=black,bg=gray!10}
\setbeamercolor{block title alerted}{fg=red,bg=gray!40}
\setbeamercolor{block title example}{fg=black,bg=green!20}
\setbeamercolor{block body example}{fg=black,bg=green!5}
\setbeamerfont{block title}{series=\bfseries}

\hypersetup{colorlinks,linkcolor=foreground,urlcolor=foreground}


\setbeamertemplate{footline}{%
    \raisebox{5pt}{\makebox[\paperwidth]{\hfill\makebox[20pt]{\color{gray}
          \scriptsize\insertframenumber}}}\hspace*{5pt}}

\addtobeamertemplate{note page}{\setlength{\parskip}{12pt}}


\newcommand{\bi}{\begin{itemize}}
\newcommand{\ei}{\end{itemize}}
\newcommand{\ig}{\includegraphics}
\newcommand{\subt}[1]{{\footnotesize \color{subtitle} {#1}}}

\let\emph\relax % there's no \RedeclareTextFontCommand
\DeclareTextFontCommand{\emph}{\bfseries\em}


\setbeamertemplate{frametitle}
{\vskip4pt
  \leavevmode
%\hbox{%
\begin{beamercolorbox}[wd=\paperwidth,ht=2ex,dp=0ex]{frametitle}%
\underline{\makebox[\paperwidth][l]{\hspace*{10pt}
\large {{\insertframetitle}}}}
\end{beamercolorbox}
%  }%
}

%\setbeamercolor{frametitle}{fg=yellow,bg=red}

\begin{document}

\AtBeginSection[]{
  \begin{frame}
  \vfill
  \centering
  \begin{beamercolorbox}[sep=8pt,center,shadow=true,rounded=true]{title}
    \underline{\makebox[0.6\paperwidth][l]{
\large {{\insertsectionhead}}}}
  \end{beamercolorbox}
  \vfill
  \end{frame}
}

\title{\large{Lecture \#16: Boosting}}
\subtitle{CS 109A, STAT 121A, AC 209A: Data Science}
\author{Pavlos Protopapas \and Kevin Rader}
%\institute{Harvard University}
\date{}
\titlegraphic{
   \includegraphics[height=2cm]{iacs}\includegraphics[height=2cm]{hogwarts}
}
{
\setbeamertemplate{footline}{} % no page number here
\frame{
  \titlepage
  
}
}


\begin{frame}{Lecture Outline}
\tableofcontents
\end{frame}

%%%%%%%%%%%%%%%%%%%%%%%%%%%%%%%%%%%%%%%%%%%%%%%%%%%%%%%%%%%%%%%%%%%%%%%%%%%%%%
\section{Review}

%%%%%%%%%%%%%%
\begin{frame}{Bags and Forests of Trees} 
\only<1>{
Last time we examined how the short-comings of single decision tree models can be overcome by ensemble methods - making one model out of many trees.
\vskip0.2cm
We focused on the problem of training large trees, these models have low bias but high variance. 
\vskip0.2cm
We compensated by training an ensemble of full decision trees and then averaging their predictions - thereby reducing the variance of our final model.
}

\only<2>{
\begin{itemize}
\item Bagging:
\begin{itemize}
\item create an ensemble of full trees, each trained on a bootstrap sample of the training set; 
\item average the predictions
\end{itemize}
\vskip0.2cm
\item Random forest:
\begin{itemize}
\item create an ensemble of full trees,each trained on a bootstrap sample of the training set; 
\item in each tree and each split, randomly select a subset of predictors, choose a predictor from this subset for splitting; 
\item average the predictions
\end{itemize}
\vskip0.2cm
Note that the ensemble building aspects of both method are embarrassingly parallel!
\end{itemize}
}

\only<3>{
But might we address the shortcomings of building single decision trees 
}
\end{frame}


%%%%%%%%%%%%%%%%%%%%%%%%%%%%%%%%%%%%%%%%%%%%%%%%%%%%%%%%%%%%%%%%%%%%%%%%%%%%%%
\section{Gradient Boosting}

%%%%%%%%%%%%%%
\begin{frame}{} 

\end{frame}

%%%%%%%%%%%%%%%%%%%%%%%%%%%%%%%%%%%%%%%%%%%%%%%%%%%%%%%%%%%%%%%%%%%%%%%%%%%%%%
\section{AdaBoost}

%%%%%%%%%%%%%%
\begin{frame}{} 

\end{frame}
\end{document}
